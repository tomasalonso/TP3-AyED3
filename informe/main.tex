\documentclass[a4paper]{article}

% ------------------------------------------------------------------------------
% Paquetes

% Input en Castellano
\usepackage[spanish]{babel}
\selectlanguage{spanish}
\usepackage[utf8]{inputenc}

% Pseudocodigo
\usepackage[linesnumbered,lined,boxed,commentsnumbered,
spanish,onelanguage]{algorithm2e}

% Símbolos matematicos
\usepackage{amsfonts, amsmath, amssymb}

% Lemas, Teoremas
\newtheorem{teorema}{Teorema}
\newtheorem{lema}{Lema}
\newtheorem{prop}{Propiedad}

% Tablas
\usepackage{booktabs}
\usepackage[table,xcdraw]{xcolor}

% Etc
\usepackage{indentfirst} % Sangría también en el primer parrafo
\usepackage[draft]{todonotes} % para poder tener notas con to-do's
\usepackage{capt-of} % Imágenes con captions.

% Simbolos Custom
% Cuando repetis mucho algo, ponelo acá

% Conjuntos
\newcommand{\Nat}[0]{\mathbb{N}} % Naturales
\newcommand{\Bin}[0]{\{0, 1\}} % Conjunto binario {0,1}

% Powerset, Conjunto de partes
\newcommand{\Pows}[1]{\mathcal{P}\left(#1\right)} % Powerset P(S)

% Complejidad
% See: https://texblog.org/2014/06/24/big-o-and-related-notations-in-latex/
\newcommand{\BigO}[1]{\mathcal{O}\left(#1\right)} %Big-O
\newcommand{\BigTheta}[1]{\Theta\left(#1\right)} % Complejidad Theta
\newcommand{\BigOmega}[1]{\Omega\left(#1\right)} % Complejidad Omega


% Links en los ítems del índice
\usepackage[colorlinks=true, linkcolor=black, citecolor=black]{hyperref}

% ------------------------------------------------------------------------------
% Caratula del DC


\usepackage{lib/caratula/caratula}
\materia{Algoritmos y Estructuras de Datos III}
\submateria{Primer Cuatrimestre de 2018}
\titulo{RTP 3}
\integrante{Alonso, Tomás}{396/16}{tomasalonso96@gmail.com}
\integrante{Bringas, Alejandra}{052/16}{abringas.t@gmail.com}
\integrante{Grosso, Alejandro}{016/16}{agrosso@dc.uba.ar}

% ------------------------------------------------------------------------------
\begin{document}

% Caratula
\maketitle

\tableofcontents
\newpage

% Secciones
\section{Introducción}
No se pueden hacer pases a bordes de la cancha de posiciones impares

\section{Modelado}

El Juego básicamente consiste en un tablero rectangular de $n \times m$ donde
$n$ es $2m$ y $m$ es impar, con un arco de tres posiciones, compiten dos equipos
de tres jugadores un partido de fútbol con un numero de jugadas determinado.


El objetivo principal es desarrollar un equipo que con una función parametrizable
que evalúa posibles jugadas siguientes y con eso decida cual es la mejor a tomar
la idea es, con las distintas heurísticas, decidir que parámetros son buenos para esa
función evaluadora y así diagramar una buena estrategia de juego general.



\subsection{Implementación}


Para implementar el juego se decidió utilizar todas estructuras de diseño propio así poder
detectar de mejor forma cualquier error que ocurriese, lo cual no fue el caso al intentar
usar lo proveído por la cátedra.
En primer lugar se decidió hacer un tablero, el mismo es el que tiene las dimensiones de la cancha y
conoce todas las reglas del juego a ser respetadas por los equipos, también tiene consigo la pelota y los
6 jugadores, el es el encargado de llevar la partida y el resultado así como también es el que tiene la
función evaluadora de jugadas que toma los genomas (los parámetros de la función dependientes de cada equipo)
y evalúa la jugada a realizar.
Luego esta el equipo, que solo tiene identificadores para sus jugadores y el dicho genoma propio de cada equipo.


Todas las estructuras que manejan una posición (jugadores y pelota) tienen en realidad, dos posiciones, actual, la que
tiene actualmente luego de la ultima jugada, y también siguiente, la posición que, de acuerdo al
movimiento elegido para evaluar, el tablero utiliza para poner la posición que va tener en la próxima jugada y así
saber como quedaría configurado el terreno para evaluarlo
El tablero usa esta variable siguiente para evaluar el puntaje de la jugada a tomar.
luego el tablero le dice los puntajes de las jugadas propuestas al equipo y este decide cual realizar, confirmandole
al tablero cual es, y este efectivamente realizando la jugada, que es básicamente actualizar la posición actual
con la siguiente.

\subsection{Algo mas?}


\section{Optimización de parámetros}
Buscar jugadas es $\BigO{1}$ debido a que la cantidad de movimientos que puede
hacer cada jugador es acotada, luego las combinaciones de movimientos posibles
son acotadas. Como mover jugadores en el tablero, en el peor caso, significa
mover todos los jugadores, más la pelota y los cálculos de disputa y de gol son
constantes, mover tablero es constante. Luego, cada paso del juego es constante.
Entonces, la complejidad de un partido varía en base al tiempo de juego, dando
una complejidad de $\BigO{\textrm{tiempo de juego}}$, a lo largo del tp
llamaremos $t$ al tiempo de juego.

\subsection{Grid-Search:}

La lógica de grid-search es plantear todo el espacio de soluciones posibles,
como este espacio es muy extenso, que en este trabajo corresponde a todos los
valores continuos de -1 a 1 que puede adoptar cada uno de los 30 genomas, no es
computacionalmente posible recorrer todo, por eso se decide empezar en algún
punto aleatorio y, con un criterio, recorrer todas las soluciones vecinas a esta
para tomar la mejor. Repitiendo el proceso hasta llegar a alguna que sea mejor que
todos sus vecinos, a este procedimiento se lo llama búsqueda local.


\subsubsection{Búsqueda Local:}


Nuestro espacio de soluciones son todas las combinaciones posibles de genes
distintos en un genoma, son 30 genes que van en los reales entre $-1$ y $1$,
entonces se comienza con un equipo aleatorio, es decir, con un genoma aleatorio,
luego se define la vecindad como todos los genomas que tiene un solo gen
cambiado en $\pm 0.1$ y saturando si llega a $-1$ o $1$ o a $0$ o $1$ en caso de
tratarse de una probabilidad. Con estos genomas se lo hace jugar secuencialmente
quedándose con el ganador siempre y, finalmente con el último ganador. Como
criterio para finalizar la búsqueda, se itera hasta que un genoma sea
mejor que todos sus vecinos, es decir, gane a todos sus vecinos.

\paragraph{Complejidad:} Debido a que el espacio de soluciones se discretizó, la
cantidad de soluciones es acotada, en particular es de 21 posibilidades por cada
genoma y 11 posibilidades para las probabilidades, lo que da un número de
$21^{27}+11^3$. Como en el peor caso se recorrería todo el espacio de busqueda,
la complejidad es $\BigO{\textrm{t}}$

\subsubsection{GRASP:}


La lógica de GRASP es simplemente tener muchas soluciones aleatorias con
búsqueda local, es decir, máximos locales y luego, con todos estos mejores de
sus vecinos, se realiza un torneo para obtener el mejor de todos esos . El
concepto se basa en que con la búsqueda local lo más probable es solo obtener un
máximo local y no el mejor de todos, entonces con GRASP la idea es aumentar la
probabilidad de obtener el mejor global o al menos obtener un buen máximo local.

El algoritmo está implementado como un ciclo de búsquedas locales.

Creemos que GRASP va a dar resultados útiles en pocas iteracions, pero que va a
ser mucho mas costoso obtener resultados mejores y realmente buenos en
comparación con los que pueda obetener el genético con el mismo tiempo de
ejecución.



\subsection{Algortimos genéticos: breve contexto}

Para la siguiente sección leímos tres publicaciones académicas que utilizan el esquema de algoritmos genéticos para optimizar parámetros relacionados con los modelados de los distintos problemas que presentan. Se aplican sobre, por ejemplo, planeamiento de almacenamientos logísticos\cite{warehouse}, agentes que aprenden a jugar al Mario\cite{mario} y modelos epidémicos que pretenden detener la proliferación de enfermedades\cite{epidemic}. Mientras leíamos estas publicaciones (y mientras decidíamos cuáles leer entre un mar de títulos interesantes) comprendimos que los algoritmos genéticos son aplicables a variadas ramas del conocimiento. Entendimos también que los resultados obtenidos dependen fuertemente de buenas decisiones a la hora de instanciar todas las partes de este esquema. Resulta imprescindible determinar una función de \emph{fitness} adecuada, que contemple las variables necesarias y ajustarla a medida que se conozcan los primeros resultados de la experimentación hasta obtener una que permita lograr buenos resultados.


Playing Mario using advanced AI techniques:

Con un algoritmo genético buscaron optimizar los pesos de una red neuronal (es como si fuera una calibración) que decide las jugadas basándose en el estado del entorno de Mario. Las salidas de esta red representan las cinco posibles teclas que pueden usarse para la siguiente jugada.
Este estado de entorno tiene 14 parámetros, que dan información sobre la poscición de los obstáculos/enemigos.
En el paper se discuten distintos parámetros a tomar en cuenta para la función de \emph{fitness} y ciertos obstáculos como la generalización del entrenamiento a más niveles del juego, pues las poblaciones del algoritmo genético iban olvidando cómo evitar enemigos anteriores una vez que llegaban a niveles más avanzados.

A Genetic Algorithm for Finding an Optimal Curing Strategy for Epidemic Spreading in Weighted Networks:

Esta publicación presenta el problema de evitar la expansión de una epidema buscando el menor costo. Se basa en modelos epidémicos modelados sobre grafos pesados, que además del uso medicinal tienen aplicaciones en redes sociales y telecomunicaciones. Por ejemplo, una epidemia en una red social podría se la difusión de una noticia falsa.

El modelo asume que un nodo puede infectarse varias veces luego de recuperarse y por teoremas previos se cuenta con que existe un valor umbral tal que si se pasa, la epidemia quedará siempre en la red. Es decir, cada nodo va a infectar a otro y cuando se recuperen van a volver infectarse indefinidamente. Este umbral se traduce en una propiedad que deben cumplir los parámetros que más adelante se explican.

Los nodos tienen una tasa o velocidad de recuperación luego de una infección, que puede ser modificada, por ejemplo, distribuyendo más vacunas a la ciudad representada por ese nodo. Tienen además un costo asociado, que puede ser el costo de distribución o de la cantidad de vacunas necesarias. Lo que se busca entonces es un conjunto de tasas de recuperación, una para cada nodo, tal que cumplan la propiedad relacionada al umbral epidémico y a la vez minimicen el costo.

Se utiliza un algoritmo genético para encontrar un buen valor para estas tasas. Muchas de las partes, como la función de cruzamiento, fueron utilizadas desde la herramienta MATLAB.

Los resultados, evaluados sobre redes reales y sintéticas, se compararon con la herramienta actual utilizada para este tipo de problemas. Se muestran en casi todos los casos costos mucho menores obtenidos con el algoritmo genético.


\subsection{Implementacion del Algoritmo Genetico:}

intro a la idea?


\subsubsection{Fitness:}


\subsubsection{Seleccion:}


\subsubsection{Crossover:}


\subsubsection{Mutaciones:}


alguna conclusion previa sobre lo que creiamos que iba a pasar


\section{Experimentación}
% Para cada generación tomamos la media y la varianza de los puntajes de todos los individuos. Creemos que es bueno tener un genoma que provenga de una generación cuya varianza sea baja, ya que esto significaría que los


Iniciamos nuestra experimentación con el algoritmo genético.
Tuvimos grandes complicaciones para experimentar con las distintas funciones de selección y crossover implementadas, pues constantemente encontramos errores o comportamientos no deseados en los partidos.
Por ejemplo, nos dimos cuenta de que se llegaba a un momento en que los jugadores maximizaban el aŕea ocupada por su equipo y estaban muy cerca del rival, pero la pelota se encontraba sola y nadie iba a buscarla.
Agregamos entonces el genoma que relaciona a los jugadores con su distancia a la pelota, para que tuvieran una mejor visión del campo de juego, y quitamos el que mide el área cubierta por los jugadores de un mismo equipo.
Una gran barrera también fue el tiempo, pues aumenta muy rápidamente al agrandar las poblaciones.

Consideramos entonces una población de 15 individuos y 30 generaciones. A esto se le agregan los parámetros 0,2 y 0,4 como fracción de selección y probabilidad de mutación respectivamente (explicadas en la Sección \ref{genetico-seleccion}).
En cuanto a parámetros del juego en sí, tenemos un tablero de $5\times10$ y tiempo 70.
Fijamos las posiciones iniciales de los jugadores de manera simétrica, ya que creemos que ubicarlos de otra manera daría diferencias entre los equipos dependiendo del lado en que comiencen.
Dada la cantidad de funciones distintas que implementamos para un mismo fin, como dos de selección de individuos, dos de crossover de genomas y dos de mutaciones, realizar todas las posibles combinaciones fue temporalmente inviable. Siguiendo el espíritu futbolísitico del trabajo, decidimos diagramar un pequeño ``torneo'' de funciones. En el mismo pensamos comparar primero las funciones de selección, observando el partido entre los equipos con esos genomas.

Notamos que si bien los equipos hacían goles y algún equipo ganaba, al invertir

% \todo[inline]{VER POR QUÉ CAMBIA EL RESULTADO DE LOS PARTIDOS DEPENDIENDO DEL LADO EN QUE EMPIECEN}

% \subsubsection*{Crossover: Bloques vs. Random (6 - 5)}

% Genoma bloque:-0.775167, -0.268792, -0.904803, 0.509839, 0.0429661, -0.749711, -0.330696, -0.72851, 0.402388, -0.690626, -0.925932, -0.548266, -0.311845, 0.474691, 0.0257145, 0.784108, -0.86784, -0.262816, 0.852761, -0.65904, 0.515403, 0.490343, 0.458956, -0.679425, -0.982449, -0.845339, 0.519684, 0.749116, 0.0456578, 0.550204, 0.0943544

% Genoma random: -0.709169, 0.36851, -0.520018, 0.6342, -0.0282138, 0.322789, -0.616418, -0.854534, -0.545993, 0.164261, 0.16543, 0.492899, -0.592142, -0.209919, -0.923623, 0.584831, 0.381569, -0.133004, -0.995532, -0.254421, 0.639068, 0.0105143, 0.494118, -0.125545, -0.821469, -0.731869, 0.141382, 0.793763, 0.388867, 0.428832, 0.170146

% Se observó que ambos equipos puedieron anotar goles de manera pareja, pero que tuvieron ciertas dificultades para robar la pelota al adversario, a pesar de cruzarlo alguna vez. Los equipos parecían tener una táctica bien definida dirigiéndose directamente al arco del oponente, aunque dependió en gran medida de estrellas, que llevaron el juego en sus espaldas. No faltó en ninguno de los dos grupos el jugador desorientado que se quedó parado en una esquina del campo de juego.
% El nivel del equipo era muy similar, pero por el tiempo de juego esta vez tenemos una victoria del crossover por bloques y se consagra ganador de las funciones de crossover.

% \subsubsection*{Selección: Cantidad vs. Puntaje (3 - 0)}

----------------------------------------------------------------------------------------------------------------------------------
\todo[inline]{Decir que nos dimos cuenta de que los equipos daban resultados distintos dependiendo de qué lado de la cancha empezaban. Pensamos que era por las posiciones en el entrenamiento: no eran simétricas. lo cambiamos y volvimos a correr.}
\todo[inline]{Creemos que área cubierta no valía la pena y lo sacamos como genoma.}
\todo[inline]{COSAS A MENCIONAR}
\begin{itemize}
    \item fijarse qué pasa con más iteraciones
    \item Notar que en algun momento, por ejemplo, cubren mucha área y están cercad e los rivales, pero la pelota quedó sola en algún lado y no se mueven para buscarla.
    \item ver que pasa con mutacionA y mutacion B
    \item notar que si hacíamos todas las combinaciones posibles de las funciones eran muchas y no se llegaba a ejecutar.
    \item nos pasa que los jugadores se quedan quietos a partir de algún momento. esto podría ser por algún error en la función evaluadora, algún bug de tablero o algúna cosa que podríamos haber medido y se nos pasó.
\end{itemize}

Tenencia de la pelota, se quedaban quietos porque no había beneficios por ir a buscar la pelota -> agregamos un gen que te da beneficio por tener la pelota

Gol siempre, si se implementaran más metricas como gol seguro, chance que la intercepten, etc. seria mejor

Decir por qué agregamos cada genoma que inicialmente no estaba

distJugador al arco, antes no había ninguna información que vinculara a los jugadores con la posición de la pelota

por ejemplo agregamos puntaje si una jugada mete gol para que siempre metan gol

Pesos negativos para sacarle penalizar una acción (para no duplicar genomas)-> agregar valores negativos al genoma

% Partidos muy variados, entre que algunos son ajustados y con muchos robos, otros se meten gol siempre. O se quedan estáticos y no se mueven más.

Forzar que si el jugador con la mínima distancia a la pelota, achica esa distancia cuando la pelota está quieta, suma puntaje

Contador de tiempo desde que la pelota se quedó quieta

% Equilibrios en el tablero donde ningún movimiento suma puntaje




\section{Conclusión}


 En principio se debe decir que se cumplieron, en mayor o menor medida, todos los objetivos
 propuestos por el enunciado y por el grupo, quizás no logramos pulir todos los aspectos
 necesarios para poder llegar a una etapa de experimentación "limpia", igualmente llegamos
 a datos coherentes.

 Es importante decir que se explica con correctitud las decisiones tomadas y el resultado final
 de la implementacion general del trabajo, desde las estructuras usadas hasta las ideas que
 definieron como se decidió utilizar las heurísticas.

 En base a todo esto, podemos decir que hubiésemos preferido orientar el trabajo más a
 las heurísticas que a las partes necesarias para que estas puedan trabajar, pero por diversas
 cuestiones encontramos mayores dificultades con esto y no fue posible.

 Aun así, logramos obtener equipos muy interesantes en la experimentación con
 ambas heurísticas y, viendo los partidos y los comportamientos de los equipos, es evidente el potencial de los métodos
 para obtener buenos parámetros en los genomas que resulten en un comportamiento que defina un equipo que sepa jugar,
  robar pelotas, meter goles e incluso realizar pases.
 El partido final entre ambas heurísticas resultó en una victoria para

 ...

 pero,
  como nuestro juego presenta algunos problemas, la comparación entre ambos no es fácil
  y no creemos que el resultado refleje con exactitud las diferencias. Pero las comparaciones
  de las distintas funciones utilizadas en el algoritmo genético si dan resultados y comportamientos
  muy interesantes de analizar.


\section{Referencias}

\begin{thebibliography}{9}
\addcontentsline{toc}{section}{References}
\bibitem{warehouse}
C. Pan, S. Yu, X. Du.
\textit{Optimization of Warehouse Layout Based on Genetic Algorithm and Simulation Technique.}
Dalian Neusoft University of Information, China. 2018.

\bibitem{mario}
L. D. Jørgensen, T. W. Sandberg.
\textit{Playing Mario using advanced AI techniques.}
2009

\bibitem{epidemic}
C. Pizzuti, A. Socievole.
\textit{A Genetic Algorithm for Finding an Optimal Curing Strategy for Epidemic Spreading in Weighted Networks.}
Inst. for High Perform. Comp. and Networking (ICAR), National Research Council of Italy (CNR). Presentado en GECCO ’18.

\bibitem{MOO}
\textit{Multi-objective optimization.}
Articulo de wikipedia en el cual se obtuvo la informacion basica de optimizacion multi-objetivo.\\
\url{https://en.wikipedia.org/wiki/Multi-objective_optimization}

\end{thebibliography}


\end{document}
