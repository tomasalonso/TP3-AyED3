\subsection{Grid-Search:}

La lógica de grid-search es plantear todo el espacio de soluciones posibles,
como este espacio es muy extenso, que en este trabajo corresponde a todos los
valores continuos de -1 a 1 que puede adoptar cada uno de los 30 genomas, no es
computacionalmente posible recorrer todo, por eso se decide empezar en algún
punto aleatorio y, con un criterio, recorrer todas las soluciones vecinas a esta
para tomar la mejor. Repitiendo el proceso hasta llegar a alguna que sea mejor que
todos sus vecinos, a este procedimiento se lo llama búsqueda local.


\subsubsection{Búsqueda Local:}


Nuestro espacio de soluciones son todas las combinaciones posibles de genes
distintos en un genoma, son 30 genes que van en los reales entre $-1$ y $1$,
entonces se comienza con un equipo aleatorio, es decir, con un genoma aleatorio,
luego se define la vecindad como todos los genomas que tiene un solo gen
cambiado en $\pm 0.1$ y saturando si llega a $-1$ o $1$ o a $0$ o $1$ en caso de
tratarse de una probabilidad. Con estos genomas se lo hace jugar secuencialmente
quedándose con el ganador siempre y, finalmente con el último ganador. Como
criterio para finalizar la búsqueda, se itera hasta que un genoma sea
mejor que todos sus vecinos, es decir, gane a todos sus vecinos.


\subsubsection{GRASP:}


La lógica de GRASP es simplemente tener muchas soluciones aleatorias con
búsqueda local, es decir, máximos locales y luego, con todos estos mejores de
sus vecinos, se realiza un torneo para obtener el mejor de todos esos . El
concepto se basa en que con la búsqueda local lo más probable es solo obtener un
máximo local y no el mejor de todos, entonces con GRASP la idea es aumentar la
probabilidad de obtener el mejor global o al menos obtener un buen máximo local.

El algoritmo está implementado como un ciclo de búsquedas locales.

Creemos que GRASP va a dar resultados útiles en pocas iteracions, pero que va a
ser mucho mas costoso obtener resultados mejores y realmente buenos en
comparación con los que pueda obetener el genético con el mismo tiempo de
ejecución.
