\subsection{Implementación del Algoritmo Genético:}

La idea del algoritmo genético se basa en la teoría de la evolución, consiste en generar una población,
en principio aleatoria, de muchos genomas, con estos se testea cuan buena solución son para el problema dado.
En este caso, como se trata de un juego, se prueba en un torneo con una función puntuadora
(\emph{Fitness}) cuales ganan mas partidos o juegan mejor, luego, en base a esto se eligen cuales van a permanecer
cuales van a reproducirse y cuales desaparecen \emph{Selección} y con los que tienen descendencia se utilizan
las funciones de \emph{Crossover} y de \emph{Mutación}.

De todo esto se va a hablar en las próximas partes:

\subsubsection{Fitness:}


La primera, \emph{fitness\_puntos} , se basa en la idea de un torneo de fútbol del mundo real, simplemente hace jugar a todos
los equipos entre si (como en una liga) y les suma $3$ puntos por ganar, $1$ por el empate y $0$ por perder.
Esperamos que esto presente un buen método de evaluación ya que se basa en algo real, y es simple en cuanto a que solo importa ganar
mas partidos, lo cual, a fin de cuentas, es lo mas importante.\\

La segunda función, \emph{fitness\_diff\_goles} trabaja de la siguiente forma, si gana le suma un punto y luego le suma tantos
puntos como diferencia de gol tenga con el contrario, por ejemplo, si gano $3$ a $1$, le suma $2$ puntos mas al obtenido por
la victoria.

Esta función intenta reforzar a los equipos que hagan muchos goles, lo cual podría llegar a indicar que juegan mucho mejor que el
contrario y podría significar que va a jugar mejor también contra otros potenciales adversarios.\\

Al final de la ejecución, ambas funciones ordenan la población de acuerdo a los puntajes para facilitar el trabajo de las funciones
de selección


\subsubsection{Selección:}


\subsubsection{Crossover:}


\subsubsection{Mutaciones:}


\todo[inline]{alguna conclusion previa sobre lo que creiamos que iba a pasar}
