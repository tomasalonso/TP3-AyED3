% Para cada generación tomamos la media y la varianza de los puntajes de todos los individuos. Creemos que es bueno tener un genoma que provenga de una generación cuya varianza sea baja, ya que esto significaría que los


Iniciamos nuestra experimentación con el algoritmo genético.
Tuvimos grandes complicaciones para experimentar con las distintas funciones de selección y crossover implementadas, pues constantemente encontramos errores o comportamientos no deseados en los partidos.
Por ejemplo, nos dimos cuenta de que se llegaba a un momento en que los jugadores maximizaban el aŕea ocupada por su equipo y estaban muy cerca del rival, pero la pelota se encontraba sola y nadie iba a buscarla.
Agregamos entonces el genoma que relaciona a los jugadores con su distancia a la pelota, para que tuvieran una mejor visión del campo de juego y quitamos el que mide el área cubierta por los jugadores de un mismo equipo.
Una gran barrera también fue el tiempo, pues aumenta muy rápidamente el al agrandar las poblaciones.

Consideramos entonces una población de 15 individuos y 30 generaciones. A esto se le agregan los parámetros 0,2 y 0,4 como fracción de selección y probabilidad de mutación respectivamente (explicadas en AGERGAR LABEL A LA SECCION).
Dada la cantidad de funciones distintas que implementamos para un mismo fin, como dos de selección de individuos, dos de crossover de genomas y dos de mutaciones, realizar todas las posibles combinaciones fue temporalmente inviable. Siguiendo el espíritu futbolísitico del trabajo, decidimos diagramar un pequeño ``torneo'' de funciones. En el mismo vamos a comparar primero las funciones de selección, observando el partido entre los equipos con esos genomas.

\todo[inline]{VER POR QUÉ CAMBIA EL RESULTADO DE LOS PARTIDOS DEPENDIENDO DEL LADO EN QUE EMPIECEN}

\subsubsection*{Crossover: Bloques vs. Random (6 - 5)}

% Genoma bloque:-0.775167, -0.268792, -0.904803, 0.509839, 0.0429661, -0.749711, -0.330696, -0.72851, 0.402388, -0.690626, -0.925932, -0.548266, -0.311845, 0.474691, 0.0257145, 0.784108, -0.86784, -0.262816, 0.852761, -0.65904, 0.515403, 0.490343, 0.458956, -0.679425, -0.982449, -0.845339, 0.519684, 0.749116, 0.0456578, 0.550204, 0.0943544

% Genoma random: -0.709169, 0.36851, -0.520018, 0.6342, -0.0282138, 0.322789, -0.616418, -0.854534, -0.545993, 0.164261, 0.16543, 0.492899, -0.592142, -0.209919, -0.923623, 0.584831, 0.381569, -0.133004, -0.995532, -0.254421, 0.639068, 0.0105143, 0.494118, -0.125545, -0.821469, -0.731869, 0.141382, 0.793763, 0.388867, 0.428832, 0.170146

Se observó que ambos equipos puedieron anotar goles de manera pareja, pero que tuvieron ciertas dificultades para robar la pelota al adversario, a pesar de cruzarlo alguna vez. Los equipos parecían tener una táctica bien definida dirigiéndose directamente al arco del oponente, aunque dependió en gran medida de estrellas, que llevaron el juego en sus espaldas. No faltó en ninguno de los dos grupos el jugador desorientado que se quedó parado en una esquina del campo de juego.
El nivel del equipo era muy similar, pero por el tiempo de juego esta vez tenemos una victoria del crossover por bloques y se consagra ganador de las funciones de crossover.

\subsubsection*{Selección: Cantidad vs. Puntaje (3 - 0)}


\todo[inline]{Decir que nos dimos cuenta de que los equipos daban resultados distintos dependiendo de qué lado de la cancha empezaban. Pensamos que era por las posiciones en el entrenamiento: no eran simétricas. lo cambiamos y volvimos a correr.}
\todo[inline]{Creemos que área cubierta no valía la pena y lo sacamos como genoma.}
\todo[inline]{COSAS A MENCIONAR}
\begin{itemize}
    \item fijarse qué pasa con más iteraciones
    \item Notar que en algun momento, por ejemplo, cubren mucha área y están cercad e los rivales, pero la pelota quedó sola en algún lado y no se mueven para buscarla.
    \item ver que pasa con mutacionA y mutacion B
    \item notar que si hacíamos todas las combinaciones posibles de las funciones eran muchas y no se llegaba a ejecutar.
    \item nos pasa que los jugadores se quedan quietos a partir de algún momento. esto podría ser por algún error en la función evaluadora, algún bug de tablero o algúna cosa que podríamos haber medido y se nos pasó.
\end{itemize}

Tenencia de la pelota, se quedaban quietos porque no había beneficios por ir a buscar la pelota -> agregamos un gen que te da beneficio por tener la pelota

Gol siempre, si se implementaran más metricas como gol seguro, chance que la intercepten, etc. seria mejor

Decir por qué agregamos cada genoma que inicialmente no estaba

distJugador al arco, antes no había ninguna información que vinculara a los jugadores con la posición de la pelota

por ejemplo agregamos puntaje si una jugada mete gol para que siempre metan gol

Pesos negativos para sacarle penalizar una acción (para no duplicar genomas)-> agregar valores negativos al genoma

% Partidos muy variados, entre que algunos son ajustados y con muchos robos, otros se meten gol siempre. O se quedan estáticos y no se mueven más.

Forzar que si el jugador con la mínima distancia a la pelota, achica esa distancia cuando la pelota está quieta, suma puntaje

Contador de tiempo desde que la pelota se quedó quieta

% Equilibrios en el tablero donde ningún movimiento suma puntaje


