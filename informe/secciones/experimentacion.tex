Iniciamos nuestra experimentación con el algoritmo genético.
\textbf{ACTUALIZAR CON LAS POBLACIONES}Tomamos poblaciones de 20 individuos, dado que a partir de ese tamaño las iteraciones se volvían demasiado costosas en tiempo. Por ese motivo, además decidimos llegar hasta 20 generaciones de nuestro algoritmo.

% Para cada generación tomamos la media y la varianza de los puntajes de todos los individuos. Creemos que es bueno tener un genoma que provenga de una generación cuya varianza sea baja, ya que esto significaría que los




Tuvimos grandes complicaciones para experimentar con las distintas funciones de seleccion y crossover implementadas, pues constantemente encontramos errores o comportamientos no deseados en los partidos.
Por ejemplo, nos dimos cuenta de que se llegaba a un momento en que los jugadores maximizaban el aŕea ocupada por su equipo y estaban muy cerca del rival, pero la pelota se encontraba sola y nadie iba a buscarla. Una gran barrera también fue el tiempo, pues aumenta muy rápidamente el requerido por cada generación si se agrandan las poblaciones.








\todo[inline]{COSAS A MENCIONAR}
\begin{itemize}
    \item fijarse qué pasa con más iteraciones
    \item Notar que en algun momento, por ejemplo, cubren mucha área y están cercad e los rivales, pero la pelota quedó sola en algún lado y no se mueven para buscarla.
    \item notar que si hacíamos todas las combinaciones posibles de las funciones eran muchas y no se llegaba a ejecutar.
    \item nos pasa que los jugadores se quedan quietos a partir de algún momento. esto podría ser por algún error en la función evaluadora, algún bug de tablero o algúna cosa que podríamos haber medido y se nos pasó.
\end{itemize}

Tenencia de la pelota, se quedaban quietos porque no había beneficios por ir a buscar la pelota -> agregamos un gen que te da beneficio por tener la pelota

Gol siempre, si se implementaran más metricas como gol seguro, chance que la intercepten, etc. seria mejor

Decir por qué agregamos cada genoma que inicialmente no estaba

distJugador al arco, antes no había ninguna información que vinculara a los jugadores con la posición de la pelota

por ejemplo agregamos puntaje si una jugada mete gol para que siempre metan gol

Pesos negativos para sacarle penalizar una acción (para no duplicar genomas)-> agregar valores negativos al genoma

Partidos muy variados, entre que algunos son ajustados y con muchos robos, otros se meten gol siempre. O se quedan estáticos y no se mueven más.

Forzar que si el jugador con la mínima distancia a la pelota, achica esa distancia cuando la pelota está quieta, suma puntaje

Contador de tiempo desde que la pelota se quedó quieta

Equilibrios en el tablero donde ningún movimiento suma puntaje


