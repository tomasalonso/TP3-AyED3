
El juego consiste en un tablero rectangular de $n \times m$ donde $n$ es mayor o
igual a $2m$ y $m$ es impar, con un arco de tres posiciones, compiten dos
equipos de tres jugadores un partido de fútbol con un número de jugadas
determinado.


El objetivo principal es desarrollar un equipo que con una función parametrizable
que evalúa posibles jugadas siguientes, decida cual es la mejor a tomar.
La idea es, con las distintas heurísticas, decidir que parámetros son buenos para esa
función evaluadora y así diagramar una buena estrategia de juego general.\\

\subsection{Implementación}


Para implementar el juego se decidió utilizar todas estructuras de diseño propio
así poder detectar de mejor forma cualquier error que ocurriese, lo cual no fue
el caso al intentar usar lo provisto por la cátedra. En primer lugar se decidió
hacer un tablero, el mismo es el que tiene las dimensiones de la cancha y conoce
todas las reglas del juego a ser respetadas por los equipos, también tiene
consigo la pelota y los 6 jugadores, el es el encargado de llevar la partida y
el resultado así como también es el que tiene la función evaluadora de jugadas
que toma los genomas (los parámetros de la función dependientes de cada equipo,
ver {\it Parametrización\/}) y evalúa la jugada a realizar. Luego está el
equipo, que solo tiene identificadores para sus jugadores y el dicho genoma
propio de cada equipo.


La operatoria designada por el equipo para obtener una jugada es la siguiente:

El equipo pide al tablero los movimientos válidos de ambos equipos y genera todas las
posibles jugadas válidas a partir de combinaciones de los movimientos válidos de
cada jugador. Como el tablero posee 2 estados, {\it actual\/} y {\it
  siguiente}, donde siguiente se obtiene a partir del estado actual luego de
aplicar una jugada, el equipo prueba cada jugada generada en el tablero,
avanzando y evalúa su puntaje, quedándose con la jugada que posea el mayor
puntaje. Dicha evaluación se realiza sobre el estado siguiente y una vez que se
tienen los movimientos definitivos de ambos equipos, se actualiza el estado
actual, que es básicamente actualizar la posición actual con la siguiente.

Debido a que evaluar en cada paso todas las posibles combinaciones de jugadas de
todos los jugadores es altamente costoso, se adoptó la siguiente estrategia. Se
mantienen quietos los jugadores propios y se busca la mejor jugada de los
contrarios; luego, se busca la mejor jugada de los jugadores propios, dada la
mejor jugada de los contrarios obtenida anteriormente.

\subsection{Parametrización}

Se decidió utilizar un genoma de 30 parámetros los cuales varían entre $-1$ y
$1$ simbolizando la importancia total que se le da a un dato del tablero, donde valores
 negativos penalizan una propiedad y positivos la benefician($-1$ si es muy malo
 que eso crezca, $0$ si es irrelevante, $1$ si es muy bueno que
crezca), por ej. representan maximizar o minimizar una distancia; también se
separa en casos, creando genomas disjuntos que representen cuando se tiene la
pelota y cuando no se posee la pelota para los mismo datos. El motivo de esto es
poder permitirle a los equipos tomar estrategias distintas cuando están atacando
o defendiendo, se presentó la idea de también permitir actuar distinto cuando se
va perdiendo o ganando pero el tiempo no permitió implementarlo y además creemos
que ya vamos a poder ver un buen comportamiento con solo esta diferenciación.


A continuación un esquema del genoma utilizado:

\begin{tabular}{r|ccc|ccc|c|c|ccc|ccc|}
\cline{2-15}
                                 & \multicolumn{6}{c|}{Distancia del}                                                                        & \multicolumn{2}{c|}{Distancia de la} & \multicolumn{6}{c|}{Distancia al}                                                                             \\
                                 & \multicolumn{6}{c|}{jugador al arco}                                                                      & \multicolumn{2}{c|}{pelota al arco}  & \multicolumn{6}{c|}{jugador rival más cercano}                                                                \\ \hline
\multicolumn{1}{|r|}{posesión}   & \multicolumn{3}{c|}{con}                            & \multicolumn{3}{c|}{sin}                            & con               & sin              & \multicolumn{3}{c|}{con}                             & \multicolumn{3}{c|}{sin}                               \\ \hline
\multicolumn{1}{|r|}{nº jugador} & 0                      & 1                      & 2 & 0                      & 1                      & 2 &                   &                  & 0                      & 1                      & 2  & 0                       & 1                       & 2  \\ \hline
\multicolumn{1}{|r|}{nº gen}     & \multicolumn{1}{c|}{0} & \multicolumn{1}{c|}{1} & 2 & \multicolumn{1}{c|}{3} & \multicolumn{1}{c|}{4} & 5 & 6                 & 7                & \multicolumn{1}{c|}{8} & \multicolumn{1}{c|}{9} & 10 & \multicolumn{1}{c|}{11} & \multicolumn{1}{c|}{12} & 13 \\ \hline
\end{tabular}

\vskip 1mm
\begin{center}
{Cuadro 1: Posiciones 0 a 13 del genoma}
\end{center}

\begin{tabular}{r|ccc|ccc|ccc|ccc|c|}
\cline{2-14}
 & \multicolumn{6}{c|}{Distancia a} & \multicolumn{6}{c|}{Distancia del jugador} & Tenencia de \\
 & \multicolumn{6}{c|}{la pelota} & \multicolumn{6}{c|}{al lateral} & la pelota \\ \hline
\multicolumn{1}{|r|}{posesión} & \multicolumn{3}{c|}{con} & \multicolumn{3}{c|}{sin} & \multicolumn{3}{c|}{con} & \multicolumn{3}{c|}{sin} &  \\ \hline
\multicolumn{1}{|r|}{nº jugador} & 0 & 1 & 2 & 0 & 1 & 2 & 0 & 1 & 2 & 0 & 1 & 2 &  \\ \hline
\multicolumn{1}{|r|}{nº gen} & \multicolumn{1}{c|}{14} & \multicolumn{1}{c|}{15} & 16 & \multicolumn{1}{c|}{17} & \multicolumn{1}{c|}{18} & 19 & \multicolumn{1}{c|}{20} & \multicolumn{1}{c|}{21} & 22 & \multicolumn{1}{c|}{23} & \multicolumn{1}{c|}{24} & 25 & 26 \\ \hline
\end{tabular}

\vskip 1mm
\begin{center}
{Cuadro 2: Posiciones 14 a 26 del genoma}
\end{center}

\begin{tabular}{r|ccc|}
\cline{2-4}
 & \multicolumn{3}{c|}{Probabilidad} \\
 & \multicolumn{3}{c|}{de quite} \\ \hline
\multicolumn{1}{|r|}{nº jugador} & 0 & 1 & 2 \\ \hline
\multicolumn{1}{|r|}{nº gen} & 27 & 28 & 29 \\ \hline
\end{tabular}

\vskip 1mm
\begin{center}
{Cuadro 3: 27 a 29 del genoma}
\end{center}

Como nota, los últimos 3 genes del genoma varían entre $0$ y $1$, es decir, no
adoptan valores negativos, ya que representan las {\it probabilidades\/} de
quite de los 3 jugadores del equipo, supusimos que esto es importante para el
juego y decidimos darle la posibilidad de variar las probabilidades a los
algoritmos de optimización.

Se incluyó en primera instancia un gen adicional que aportaba información sobre
el área cubierta por el equipo, sin embargo,  fue descartado debido a que
generaba equipos que se dispersaban por el campo de juego y al quedar la pelota
quieta no evaluaban como beneficioso acercarse a buscarla. Al retirar el gen,
se encontró más variedad de comportamientos en los equipos. Una mayor cantidad de
genes e iteraciones de los algoritmos ayudaría a equilibrar este valor, pero
debido al mal comportamiento exhibido, se descartó.

El tablero posee una función {\it puntaje\/} que toma un genoma y
multiplica cada valor del genoma por el dato con el que está relacionado,
dependiendo si el equipo esta en posesión de la pelota o no. Por ejemplo,
multiplica la distancia de un jugador al arco, al rival, la distancia de la
pelota al arco contrario, la distancia a los borde, el área cubierta y si el
siguiente paso es gol o no, cada uno con su respectivo gen.



Como consideración adicional, si bien los genes se mueven en un espacio continuo entre $-1$ y $1$,
lo cual significa que cualquier peso otorgado al valor del dato no deberia generar
cambios de valor en los resultados finales sino solo de cantidad de puntos,
nosotros en grid search, para aplicar busqueda local, discretizamos el valor de los
genes haciendolos variar en $0,1$ por vez.


A causa de esto se decidió se agregar una multiplicación por $100$ en cada
puntaje, para poder disminuir el efecto de esa discretizacion en el resultado final.

% Si bien los genes son de valores continuos y los algoritmos genéticos y
% de búsqueda local, teóricamente encontrarían

Por último, se tomó la decisión de otorgar puntaje a un movimiento que termina
en gol y restar mucho puntaje en caso de gol en contra para beneficiar y
prohibir dichos movimientos respectivos.
