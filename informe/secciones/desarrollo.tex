
El Juego básicamente consiste en un tablero rectangular de $n \times m$ donde
$n$ es $2m$ y $m$ es impar, con un arco de tres posiciones, compiten dos equipos
de tres jugadores un partido de fútbol con un numero de jugadas determinado.


El objetivo principal es desarrollar un equipo que con una función parametrizable
que evalúa posibles jugadas siguientes y con eso decida cual es la mejor a tomar
la idea es, con las distintas heurísticas, decidir que parámetros son buenos para esa
función evaluadora y así diagramar una buena estrategia de juego general.



\subsection{Implementación}


Para implementar el juego se decidió utilizar todas estructuras de diseño propio así poder
detectar de mejor forma cualquier error que ocurriese, lo cual no fue el caso al intentar
usar lo proveído por la cátedra.
En primer lugar se decidió hacer un tablero, el mismo es el que tiene las dimensiones de la cancha y
conoce todas las reglas del juego a ser respetadas por los equipos, también tiene consigo la pelota y los
6 jugadores, el es el encargado de llevar la partida y el resultado así como también es el que tiene la
función evaluadora de jugadas que toma los genomas (los parámetros de la función dependientes de cada equipo)
y evalúa la jugada a realizar.
Luego esta el equipo, que solo tiene identificadores para sus jugadores y el dicho genoma propio de cada equipo.


Todas las estructuras que manejan una posición (jugadores y pelota) tienen en realidad, dos posiciones, actual, la que
tiene actualmente luego de la ultima jugada, y también siguiente, la posición que, de acuerdo al
movimiento elegido para evaluar, el tablero utiliza para poner la posición que va tener en la próxima jugada y así
saber como quedaría configurado el terreno para evaluarlo
El tablero usa esta variable siguiente para evaluar el puntaje de la jugada a tomar.
luego el tablero le dice los puntajes de las jugadas propuestas al equipo y este decide cual realizar, confirmandole
al tablero cual es, y este efectivamente realizando la jugada, que es básicamente actualizar la posición actual
con la siguiente.

\subsection{Algo mas?}
