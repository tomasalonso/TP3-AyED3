
En este trabajo práctico se va a investigar y explorar la resolución de problemas con
diversas heurísticas de exploración de soluciones, en principal se va a trabajar
con búsqueda local, GRASP y algoritmos genéticos. Particularmente se va a intentar trabajar
con la toma de decisiones estratégicas en un juego de tipo adversarial.\\


En principio se pretende modelar un partido de fútbol donde dos equipos de 3
jugadores compiten para ver cuál de los dos obtiene más goles en un período de tiempo. El
equipo con mayor cantidad de goles será el ganador. En caso de que ambos equipos tengan
la misma cantidad de goles realizados, el resultado será un empate. Nuestro objetivo es lograr
conseguir un equipo que tenga una buena estrategia de juego con las mencionadas técnicas
algorítmicas. El objetivo es, en primer lugar, modelar computacionalmente el fenómeno descrito, luego
parametrizar propiedades de dicho modelo que sean controlables e implementar, con distintas técnicas
algorítmicas, algoritmos que permitan conseguir buenos valores para dichos parámetros. Estos parámetros
serán los generadores de distintas soluciones al problema modelado.
\\


Se procederá así:
En la próxima sección se describirá el modelo lógico y luego, con más detalle, cómo fue implementado
computacionalmente el mismo,
así como también una explicación de cada decisión tomada ya sea por necesidad o facilidad para
la aplicación de las heurísticas sobre esta implementación. El objetivo es que una solución
para el modelo sea también una solución para el problema real.
\\


Luego se explicará el razonamiento y las decisiones tomadas en cuanto a las ideas desarrolladas para
la resolución del problema. Se mostrara cómo funciona la heurística tratada en particular,
daremos también ejemplos del funcionamiento y se explicará en detalle por qué es
efectivamente una implementación de la técnica elegida, hablaremos también de qué esperamos notar
en los resultados y de la complejidad temporal de cada algoritmo.
\\


Por último vamos a realizar una experimentación computacional para medir la performance de
los algoritmos implementados así como sus distintas propiedades. Para ello vamos a trabajar con
todas las heurísticas al mismo tiempo, comparando sus resultados para corridas del aproximadamente
el mismo costo temporal. Luego se van a desarrollar experimentos que permiten medir los aspectos
evolutivos de las soluciones obtenidas mediante las distintas técnicas utilizadas en
el trabajo para poder compararlas entre sí en base a estos aspectos.\\
