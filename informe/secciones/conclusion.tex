En este trabajo propusimos un modelado del juego planteado en el enunciado y dos heurísticas para optimizar sus parámetros.
El modelado nos tomó mucho tiempo por su complejidad.
Hubiéramos preferido dedicarle más tiempo a realizar distintas funciones de fitness, por ejemplo, pero el hecho de encontrar errores en el juego nos hizo volver una y otra vez al código base.

A pesar de no haber conseguido equipos que jueguen de manera profesional, consideramos un gran logro el hecho de que haya marcas, quites, pases y goles.
Viendo los partidos y los comportamientos de los equipos, es evidente el potencial de los métodos para obtener buenos parámetros en los genomas.
Como nuestro juego presenta algunos problemas, la comparación entre dos equipos no es fácil y no creemos que el resultado refleje con exactitud las diferencias. Pero las comparaciones de las distintas funciones utilizadas en el algoritmo genético sí dan resultados y comportamientos interesantes de analizar.
Igualmente encontramos ciertas tendencias en algunos de los genes que se corresonden con cierta manera de jugar.
Nos hubiera gustado experimentar con poblaciones más grandes y más generaciones, que tal vez hubieran logrado un resultado mejor.

Nos quedó pendiente encontrar la razón detrás de la diferencia de jugadas si se empieza de un lado u otro.

Podríamos haber aprovechado la ayuda del internacionalmente reconocido genetista y docente de la Facultad, Dr. Alberto Kornblihtt (Ver Anexo-\ref{ANEXO}).














 % En principio se debe decir que se cumplieron, en mayor o menor medida, todos los objetivos
 % propuestos por el enunciado y por el grupo, quizás no logramos pulir todos los aspectos
 % necesarios para poder llegar a una etapa de experimentación "limpia", igualmente llegamos
 % a datos coherentes.

 % Es importante decir que se explica con correctitud las decisiones tomadas y el resultado final
 % de la implementacion general del trabajo, desde las estructuras usadas hasta las ideas que
 % definieron como se decidió utilizar las heurísticas.

 % En base a todo esto, podemos decir que hubiésemos preferido orientar el trabajo más a
 % las heurísticas que a las partes necesarias para que estas puedan trabajar, pero por diversas
 % cuestiones encontramos mayores dificultades con esto y no fue posible.

 % El partido final entre ambas heurísticas resultó en una victoria para

 % ...

 % pero,
 %  como nuestro juego presenta algunos problemas, la comparación entre ambos no es fácil
 %  y no creemos que el resultado refleje con exactitud las diferencias. Pero las comparaciones
 %  de las distintas funciones utilizadas en el algoritmo genético si dan resultados y comportamientos
 %  muy interesantes de analizar.
