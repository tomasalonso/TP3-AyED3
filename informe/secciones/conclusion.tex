

 En principio se debe decir que se cumplieron, en mayor o menor medida, todos los objetivos
 propuestos por el enunciado y por el grupo, quizás no logramos pulir todos los aspectos
 necesarios para poder llegar a una etapa de experimentación "limpia", igualmente llegamos
 a datos coherentes.

 Es importante decir que se explica con correctitud las decisiones tomadas y el resultado final
 de la implementacion general del trabajo, desde las estructuras usadas hasta las ideas que
 definieron como se decidió utilizar las heurísticas.

 En base a todo esto, podemos decir que hubiésemos preferido orientar el trabajo más a
 las heurísticas que a las partes necesarias para que estas puedan trabajar, pero por diversas
 cuestiones encontramos mayores dificultades con esto y no fue posible.

 Aun así, logramos obtener equipos muy interesantes en la experimentación con
 ambas heurísticas y, viendo los partidos y los comportamientos de los equipos, es evidente el potencial de los métodos
 para obtener buenos parámetros en los genomas que resulten en un comportamiento que defina un equipo que sepa jugar,
  robar pelotas, meter goles e incluso realizar pases.
 El partido final entre ambas heurísticas resultó en una victoria para

 ...

 pero,
  como nuestro juego presenta algunos problemas, la comparación entre ambos no es fácil
  y no creemos que el resultado refleje con exactitud las diferencias. Pero las comparaciones
  de las distintas funciones utilizadas en el algoritmo genético si dan resultados y comportamientos
  muy interesantes de analizar.
